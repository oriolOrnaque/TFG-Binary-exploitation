\documentclass[../tfg.tex]{subfile}

\begin{document}

\thispagestyle{plain}
\begin{center}
    \Large
    \textbf{Binary exploitation}

    \vspace{0.4cm}
    \large
    Memory corruption

    \vspace{0.4cm}
    \textbf{Oriol Ornaque Blázquez}

    \vspace{0.9cm}
    \textbf{Abstract}
\end{center}

Binaries, or programs compiled down to executables, might come with errors or bugs that could trigger behavior unintended by their authors. By carefully understanding the environment where programs get executed, the instructions and the memory, an attacker can gracefully craft a specific input, tailored to trigger these unintended behaviors and gain control over the original logic of the program. One of the ways this could be achieved, is by corrupting critical values in memory.

This works focuses on the main techniques to exploit buffer overflows and other memory corruption vulnerabilities to exploit binaries. Also a proof-of-concept for CVE-2021-3156 is presented with an analysis of its inner workings.

\begin{center}
    \large
    \vspace{0.9cm}
    \textbf{Resum}
\end{center}

Els binaris, o programes compilats en executables, poden venir amb errors o bugs que podrien desencadenar un comportament no previst pels seus autors. Entendre acuradament l'entorn en el qual s'executen els programes, les instruccions i la memòria, permet a un atacant elaborar dades d'entrada específiques, adaptades per desencadenar aquests comportaments no desitjats i obtenir el control sobre la lògica original del programa. Una de les maneres d'aconseguir-ho és corrompent valors crítics en la memòria del programa.

Aquest treball es centra en les principals tècniques per explotar desbordaments de memòria i altres vulnerabilitats de corrupció de memòria per a explotar binaris. També es presenta una prova de concepte, una demostració, de CVE-2021-3156 amb una anàlisi del seu funcionament.

\begin{center}
    \large
    \vspace{0.9cm}
    \textbf{Resumen}
\end{center}

Los binarios, o programas compilados en ejecutables, pueden venir con errores o bugs que podrían desencadenar un comportamiento no previsto por sus autores. Al entender cuidadosamente el entorno en el que se ejecutan los programas, las instrucciones y la memoria, un atacante puede elaborar datos de entrada específicos, adaptados para desencadenas estos comportamientos no deseados y obtener el control sobre la lógica original del programa. Una de las formas de conseguirlo es corrompiendo valores críticos en la memoria del programa.

Este trabajo se centra en las principales técnicas para explotar desbordamientos de memoria y otras vulnerabilidades de corrupción de memoria para explotar binarios. También se presenta una prueba de concepto, una demostración, de CVE-2021-3156 con un análisis de su funcionamiento.

\end{document}

