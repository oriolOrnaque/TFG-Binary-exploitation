\documentclass[../tfg.tex]{subfile}

\begin{document}

\section{CVE Program}
The Common Vulnerability and Exposures program is a system to identify and classify publicly disclosed cybersecurity vulnerabilities. It is a database holding records for each vulnerability identified and disclosed by researches and organizations. CVE records are used to ensure common definitions for issues.

\subsection{CVE IDs}
A CVE ID is the unique identifier used to refer to a vulnerability on the CVE database. The identifier follows as a format the CVE prefix, followed by the year of the registration, ended by arbitrary digits.
For example, the CVE ID of the vulnerability showed on the Chapter \ref{example:baron_samedit} nicknamed "Baron Samedit" by its authors at Qualys is \texttt{CVE-2021-3156}: being 2021 the year of registration.

\subsection{CNAs}
CVEs are assigned by CVE Numbering Authorities, being them formal and partnered organizations with the CVE Program. When a researcher or organization finds some vulnerability they request a CNA to assign a CVE ID to the vulnerability and registers it to the CVE database as a record \textbf{only} if the vendor or owner of the software allows to publicly disclose the vulnerability.

\section{CWE Program}
The Common Weakness Enumeration is a public database of common software and hardware weakness types that could lead to security issues. The goal is to educate software and hardware engineers on how to stop vulnerabilities at the source by identifying and classifying these weaknesses in a taxonomy. The records of the database also contain examples, related weaknesses, consequences, mitigations, among other things to help the users preventing vulnerabilities.

\end{document}

